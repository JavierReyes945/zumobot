\documentclass{article}

\title{Zumobot case study}
\date{29.6.2017}
\author{Javier Reyes}

\usepackage{graphicx}
\usepackage{booktabs}

\usepackage[backend=biber, style=numeric]{biblatex}
\bibliography{base}


\begin{document}

\maketitle
\pagenumbering{gobble}
\newpage
\pagenumbering{arabic}

\tableofcontents
\newpage

\section{Introduction}

One of the most used study cases in Control Theory is the Inverted Pendulum analysis, as it present an unstable open-loop characteristic but is also possible to stabilize on a closed-loop configuration.

Here we present the anyalis of the Zumobot 32U4, a small robot available on the market. It is equiped with two DC motors, an Arduino-based board with several periferials.

\section{System analysis}

An inverted pendulum can be represented as a cart moving on an horizontal axis, connected to a rigid body pendulum.

\begin{figure}[h!]
	\includegraphics{img/zumo-superior.png}
	\centering
	\caption{Superior view of the zumo robot.}
	\label{fig:sup-zumo}
\end{figure}

The figure \ref{fig:sup-zumo} shows the physical representation of the elements to be part of the analysis.

\begin{table}[h!]
	\centering
	\caption{Basic table.}
	\label{tab:tab1}
	\begin{tabular}{ccc}

		\toprule

		header A & header B & header C\\

		\midrule

		a & b & c\\
		c & d & e\\
		f & g & h\\

		\bottomrule

	\end{tabular}
\end{table}

Text of the section\footnote{\label{fn1}This is a footnote}.

Text of the section after the footnote \ref{fn1}.

\section{Mathematical development}

Text of the section\cite{DOE17} with a reference.

Text of the section\cite{TYSON08} with another reference.

\section{System model}

Text of the section.

\section{Regulator}

Text of the section.

\section{Implementation}

Text of the section.

\begin{appendix}
	\newpage
	\listoffigures
	\newpage
	\listoftables
\end{appendix}

\newpage
\printbibliography

\end{document}
