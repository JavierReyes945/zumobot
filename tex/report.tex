\documentclass{article}

\title{Zumobot case study}
\date{29.6.2017}
\author{Javier Reyes\inst{1} \and Hari Kumar\inst{2}}
\institute[Fachhochschule Dortmund]
{
  \inst{1}
  Faculty of Electronics and Informatic\\
  Fachhochschule Dortmund
  \and
  \inst{2}
  Faculty of Electronics and Informatic\\
  Fachhochschule Dortmund}

\usepackage{graphicx}
\usepackage{booktabs}
\usepackage{amsmath}
\usepackage{comment}

\usepackage[backend=biber, style=ieee]{biblatex}
\bibliography{report}


\begin{document}

\maketitle
\pagenumbering{gobble}
\newpage
\pagenumbering{arabic}

\tableofcontents
\newpage

\section*{Introduction}

One of the most used study cases in Control Theory is the Inverted Pendulum analysis, as it present an unstable open-loop characteristic but is also possible to stabilize on a closed-loop configuration.

Here we present the anyalis of the Zumobot 32U4, a small robot available on the market. It is equiped with two DC motors, an Arduino-based board with several periferials.

TODO: Complete

\section{System analysis}

An inverted pendulum can be represented as a cart moving on an horizontal axis, connected to a rigid body pendulum.

\begin{comment}
	\begin{figure}[!htbp]
		\includegraphics{img/zumo-superior.png}
		\centering
		\caption{Superior view of the zumo robot.}
		\label{fig:sup-zumo}
	\end{figure}
\end{comment}

%The figure \ref{fig:sup-zumo} shows the robot that will be modeled as a Wheeled Inverted Pendulum.

\begin{comment}
	\begin{table}[!htbp]
		\centering
		\caption{Basic table.}
		\label{tab:tab1}
		\begin{tabular}{ccc}

			\toprule

			header A & header B & header C\\

			\midrule

			a & b & c\\
			c & d & e\\
			f & g & h\\

			\bottomrule

		\end{tabular}
	\end{table}
\end{comment}

Text of the section\footnote{\label{fn1}This is a footnote}.

Text of the section after the footnote \ref{fn1}.

TODO: Complete

\section{Mathematical development}

Through the literature is possible to find several approaches to obtain a model for the IP.

To fully model the dynamic behavior of the zumo robot, we need to consider the equations that govern the movement as rigid body.

\subsection{First approach - Sum of forces}

The first approach considered here is described in \cite{SUL03}, as follows:

The equations of motion are obtained from the sum of forces in the cart for the horizontal direction.

\begin{equation} \label{sfch}
F-b\cdot \dot{x}-N=M\cdot \ddot{x}
\end{equation}

Now considering the pendulum itself, the force applied in the horizontal direction due to the momentum of the pendulum is determined as:

\begin{equation} \label{dhfp}
\tau=r\cdot F=I\cdot \ddot{\theta}
\end{equation}

Given the fact that the moment of inertia of a pendulum of mass $m$ is defined as $I=m\cdot L^2$, the previous equation can be rewritten as:

\begin{equation} \label{dhfp2}
F=\frac{I\cdot \ddot{\theta}}{r}=\frac{m\cdot l^2\cdot \ddot{\theta}}{l}=m\cdot l\cdot \ddot{\theta}
\end{equation}

Obtaining the component of the force defined in \ref{dhfp2} in the horizontal direction:

\begin{equation} \label{sfph}
F=m\cdot l\cdot \ddot{\theta}\cdot \cos{\theta}
\end{equation}

Now, the component of the centripetal force acting on the pendulum is similar to the one in \ref{dhfp2}, but the horizontal component of this force is:

\begin{equation} \label{cfph}
F=m\cdot l\cdot \dot{\theta}^2\cdot \sin{\theta}
\end{equation}

Summing the defined forces present in the horizontal direction of the pendulum in \ref{sfph} and \ref{cfph}, we obtain the following expression:

\begin{equation} \label{nfp}
N=m\cdot \ddot{x}+m\cdot l\cdot \ddot{\theta}\cdot \cos{\theta}-m\cdot l\cdot \dot{\theta}^2\cdot \sin{\theta}
\end{equation}

Now we can substitute \ref{nfp} into \ref{sfch}, we obtain the first equation of motion:

\begin{equation} \label{fem}
F=(M+m)\ddot{x}+b\cdot \dot{x}+m\cdot l\cdot \ddot{\theta}\cdot \cos{\theta}-m\cdot l\cdot \dot{\theta}^2\cdot \sin{\theta}
\end{equation}

\section{System model}

Text of the section.

\section{Regulator}

Text of the section.

\section{Implementation}

Text of the section.

\begin{appendix}
	\newpage
	\listoffigures
	\newpage
	\listoftables
\end{appendix}

\newpage
\printbibliography

\end{document}
